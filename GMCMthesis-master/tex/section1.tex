\section{问题重述}

\subsection{引言}

创意平板折叠桌注重于表达木制品的优雅和设计师所想要强调的自动化与功能性。为了增大有效使用面积。设计师以长方形木板的宽为直径截取了一个圆形作为桌面,又将木板剩余的面积切割成了若干个长短不一的木条,每根木条的长度为平板宽到圆上一点的距离,分别用两根钢筋贯穿两侧的木条,使用者只需提起木板的两侧,便可以在重力的作用下达到自动升起的效果,相互对称的木条宛如下垂的桌布,精密的制作工艺配以质朴的木材,让这件工艺品看起来就像是工业革命时期的机器。

\subsection{问题的提出}

\subsubsection{问题一}

围绕创意平板折叠桌的动态变化过程、设计加工参数,本文依次提出如下问题:

(1)给定长方形平板尺寸 ($120 cm \times 50 cm \times 3 cm$),每根木条宽度(2.5 cm),连接桌腿木条的钢筋的位置,折叠后桌子的高度(53 cm)。要求建立模型描述此折叠桌的动态变化过程,并在此基础上给出此折叠桌的设计加工参数和桌脚边缘线的数学描述。



(2)折叠桌的设计应做到产品稳固性好、加工方便、用材最少。对于任意给定的折叠桌高度和圆形桌面直径的设计要求,讨论长方形平板材料和折叠桌的最优设计加工参数,例如,平板尺寸、钢筋位置、开槽长度等。对于桌高70 cm,桌面直径80 cm的情形,确定最优设计加工参数。


(3)给出软件设计的数学模型,可以根据客户任意设定的折叠桌高度、桌面边缘线的形状大小和桌脚边缘线的大致形状,给出所需平板材料的形状尺寸和切实可行的最优设计加工参数,使得生产的折叠桌尽可能接近客户所期望的形状,并根据所建立的模型给出几个设计的创意平板折叠桌。要求给出相应的设计加工参数,画出至少8张动态变化过程的示意图。